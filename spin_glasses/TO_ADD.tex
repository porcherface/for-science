\begin{document}

\subsection{Dynamic arrest in simulations and parallel tempering}

By looking the relaxation curves for the replica overlaps of the system, we can see that below the dynamical temperature dynamic arrest occurs. That is, the overlap values do not go to zero, while the equilibrium properties at this temperature suggest us thatthe overlap distribution should be $P(Q) = \delta(Q-0)$.

IMAGE MISSING!!!!

However, there is a possible workaround to estimate the correct overlap distribution at equilibrium, and is given by parallel tempering. 

Parallel tempering is a simluation technique invented to deal with problems involving multi-basin sampling.
The idea is to evolve a certain number of 'replicas' (techincally these are not replicas of the system, since they are at higher temperatures) at temperatures higher that the sampling ones.
HIgher temperatures allows those probes to sample the energy landascape without being freezed in a single basin.
When a high temperature probe finds a basin that has similar energy to the real system, the simulation allows the exchange of temperatures, letting the cold probe to sample the new basin, and turning the old sampler into an high temperature probe. By means of this impementation, parallel tempering is usually also referred as replica exchange method.

The detail of the algorithm is very similar to the one used in the Metropolis algorithm. When a certain number of montecarlo sweeps have been made on the system, replica exchange picks two configurations at random, $\{S_a,\beta_a\},\{S_b,\beta_b\}$, and evaluates the probabilty of the move, according to Metropolis Hastings algorithm. 

\begin{equation}
p = \min ( 1, { \exp( -beta_a E[S_b] - beta_b E[S_a]) \over \exp( -beta_a E[S_a] - beta_b E[S_b])   } )
\end{equation}

The ratio of the two exponential may be re-evaluated with some simple algebra to obtain

\begin{equation}
p = \min ( 1, { \exp(  \Delta beta \Delta E   } ) 
\end{equation}

It is easy to see that this condition satisfies detailed balance.

IMAGE MISSING

In this way we should be able to recover the static properties of the system under the condition of dynamical arrest.

\end{document}