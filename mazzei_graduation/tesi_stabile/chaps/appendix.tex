\chapter{Boundaries of the Cayley tree}

Consider a Cayley tree of coordination number $K$. The number of site on a given shell $s$ can be evaluated as follows:

\begin{eqnarray}
n_{s} &=& Kn_{s-1} \nonumber\\
      &=& K^2n_{s-2} \nonumber\\
		&=& ... \nonumber\\
		&=& K^{s-1}n_1 \nonumber\\
		&=& K^{s-1}(K+1)n_0 \nonumber\\
\end{eqnarray}

where $n_0=1$ and represents the seed of the lattice. The fraction of sites lying on the boundary is evaluated using

\begin{eqnarray}
\lim_{s\rightarrow\infty} \frac{n_{s}}{\sum_{j=0}^{s-1} n_{j}}
\end{eqnarray}

The sum on the denominator of the equation is easy to determine. If we regroup the $(K+1)$ common factor we may evaluate the sum as a truncated geometric series. It is well know that a sum of the type $\sum_{j=0}^n x^j$ can be estimated as $ {1-x^{n+1} \over 1 - x} $. Thus it is possible to write (remembering that each $n_j$ contains a term proportional to $K^j-1$ and the last term has $j=s-1$, and neglecting the first term of the series when extracting the common term $K+1$)

\begin{eqnarray}
\lim_{s\rightarrow\infty} \frac{n_{s}}{\sum_{j=0}^{s-1} n_{j}} &=&  {(K+1)K^{s-1}\over
{(K+1)(1-K^{s-1}) \over (K+1)}} \nonumber \\
                                                               &=&  {K^{s-1}(1-K)  \over 1-K^{s-1}} \nonumber \\
                                                               &=&   K \nonumber
\end{eqnarray}

Thus the for each site in the internal region of the Cayley tree we may find $K$ sites on the frontier, in the thermodynamic limit.

%\chapter{Evaluation of the x-derivative of the free energy}

%Before starting we remind that the total free energy of the BLSG may be written as $F = {K+1 \over 2}F_l - K F_s$, where
%$F_l$ and $F_s$ are respectively given by
%
%\begin{eqnarray}
%-\beta F_s &=& E_J\langle \ln \sum_{\alpha} W^\alpha \exp[-\beta \Delta F_s(J_1\ldots,h_1\ldots,h_{K+1})] \rangle \nonumber \\
%-\beta F_l &=& E_JE_{L}\langle \ln \sum_{\alpha} W^\alpha \exp[-\beta \Delta F_l(J,J_1\ldots,h_1\ldots,L_1\ldots,g_1\ldots,g_{K},)] \rangle \nonumber
%\end{eqnarray}%

%The bond contribution is evaluated using the free energy shift obtained when adding a new cavity
%bond to the system (the variable $J$ without any index is the cavity bond).
%If we perform the x-derivative of $F(x)$ we obtain
%


\chapter{Weighted sums with exponential density}

Here is a list of two theorems useful to evaluate weighted averages in the RSB scheme.

\begin{theorem}

Consider a set of $M$ iid random free energies, distributed with the exponential density $\rho(f) \propto \exp(\beta x (f-f_R))$. Then, neglecting terms which go to
zero when $M$ goes to infinity, the following relation holds:

\begin{equation}
\langle \ln \frac{ \sum_k a_k \exp(-\beta f_k) }{\sum_k \exp(-\beta f_k)}\rangle = \frac{1}{x} \ln (\frac{1}{M} \sum_k a_{k}^x)
\label{app1}
\end{equation}
\end{theorem}

\begin{corollary}
With the same conditions of the previous theorem, the following relation is valid.

\begin{equation}
\langle\frac{ \sum_k a_k b_k\exp(-\beta f_k) }{\sum_k a_k\exp(-\beta f_k)}\rangle = \frac{ \sum_k a_{k}^x b_k}{\sum_k a_{k}^x}
\label{app2}
\end{equation}
\end{corollary}

As it is easy to see, once \ref{app1} is demonstrated, \ref{app2} follows in a straightforward way.


\chapter{Algorithm core}
\section{Main part}
The main file of the two step RSB algorithm is reported. The main file performs a simulation for each pair of the values in the x_rsb_sample and y_rsb_sample arrays (called $x_s$ and $x_C$ in the discussion).

In the first phase of the study the goal was to find the couple that maximizes $F$, so the range of the couple $(x_s,x_C)$ was in this way easy to configure.
When the correct couple has been found, the same procedure shall be used to do estimate the error
in the evaluation of each observable.

\begin{verbatim}

#include "defines.h"

int main(){

	//be polite	
	cout << "hello!";


	//random seed for randStream
	srand(0);

	//remove previous data
	remove(FILE_NAME_XY);

	//initialize variables
	mySystem Sys;

	//init del sistema
	init();


	//////////////////////////////////////////////////////////////////////////////////////////
	//////////////////////////////////////////////////////////////////////////////////////////
	Sys.time = 0;
	while (Sys.time < MAX_TIME)
	{

		//choose indexes
		Sys.currSiteIndex = Sys.time%Nh;
		Sys.currSite = &Sys.Site[Sys.currSiteIndex];

		Sys.choose_sites();

		//merge
		Sys.merge_fields();

		//reweigh
		Sys.currSite->reweigh();

		//evaluate free energy
		Sys.currSite->evaluate();

		if (Sys.time > BIAS && Sys.time % Nh == 0){

			//observables
			Sys.evaluate_observables();
			Sys.write_file();
		}

		if (Sys.time == MAX_TIME / 2){
			cout << "Halfway done ";
		}
		if (Sys.time % (MAX_TIME / 10) == 0){
			cout << ".";
		}

		Sys.time++;

	}//endsim

	return 0;
}



\end{verbatim}

\section{Merge procedure and free energy shift}

In this appendix I will provide the routine used when merging $K$ branches into a site. In this code I will also evaluate the free energy shift, at state level.

\begin{verbatim}
void mySystem::merge_fields(){

	// evaluate for each cluster & state the
	// merged field and the free energy
	double u, usite, ulink;
	double beta = 1. / T;
	double shift,shiftsite,shiftlink1,shiftlink2;

	double h0,h0site,h0link,g0link;

	for (int cit = 0; cit < Nc; cit++){
		for (int sit = 0; sit < Nc; sit++)
		{
			h0 = 0;
			h0site = 0;
			h0link=0;
			g0link = 0;
			shift = 0;
			shiftsite = 0;
			shiftlink1 = 0;
			shiftlink2 = 0;
			for (int k = 0; k < 2 * K; k++){
				if (k < K){
					u = T*atanh(tanh(beta*Jiter[k])*
						tanh(beta*Site[iterNeighs[k]].Cluster[cit].State[sit]));
					h0 += u;
					shift += log(cosh(beta*Jiter[k]) / cosh(beta*u));
				}
				if (k < K + 1){
					usite = T*atanh(tanh(beta*Jsite[k])*
						tanh(beta*Site[siteNeighs[k]].Cluster[cit].State[sit]));
					shiftsite += log(cosh(beta*Jsite[k]) / cosh(beta*usite));
					h0site += usite;
				}
				if (k < K){
					ulink = T*atanh(tanh(beta*Jlink[k])*
						tanh(beta*Site[linkNeighs[k]].Cluster[cit].State[sit]));
					h0link += ulink;
					shiftlink1 += log(cosh(beta*Jlink[k]) / cosh(beta*ulink));
				}
				else{
					ulink = T*atanh(tanh(beta*Jlink[k])*
						tanh(beta*Site[linkNeighs[k]].Cluster[cit].State[sit]));
					g0link += ulink;
					shiftlink2 += log(cosh(beta*Jlink[k]) / cosh(beta*ulink));
					
				}
			}
			//add low noise
			
			double noise = 0.0001;

			h0 += noise*((double)rand() / RAND_MAX - 0.5);
			h0site += noise*((double)rand() / RAND_MAX - 0.5);
			h0link += noise*((double)rand() / RAND_MAX - 0.5);
			g0link += noise*((double)rand() / RAND_MAX - 0.5);

			double J0 = ((rand())>.5*RAND_MAX) * 2 - 1;

			double v0 = T*atanh((tanh(beta*J0) + tanh(beta*h0link)*tanh(beta*g0link)) /
				(1 + tanh(beta*J0)*tanh(beta*h0link)*tanh(beta*g0link)));

			currSite->Cluster[cit].State[sit] = h0;
			currSite->Cluster[cit].siteState[sit] = h0site;
			currSite->Cluster[cit].linkState[sit] = v0;

			double newlinkshift = 0;
			double expo=0;
			double sigma, tau;

			
			for (int tau = -1; tau <= 1; tau = tau + 2){
				for (int sigma = -1; sigma <= 1; sigma = sigma + 2){
					expo += exp(beta*J0*sigma*tau + beta*sigma*h0link + beta*tau*g0link);
				}
			}

			newlinkshift = log(expo);

			currSite->Cluster[cit].dFiter[sit] = shift + log(2 * cosh(beta*h0));
			currSite->Cluster[cit].dFsite[sit] = shift + log(2 * cosh(beta*h0site));
			currSite->Cluster[cit].dFlink[sit] = shiftlink1 + shiftlink2 + newlinkshift;
		}

	}

}
\end{verbatim}

It is possible to note that in the last lines the weights are transformed into their cumulative, this will be used in the following routine.

\section{The reweigh procedure}

The simplest version of the reweigh is reported.

\begin{verbatim}
void myCluster::reweigh()
{


	// get weights
	for (int it = 0; it < Ns; it++){
		cumulativeIter[it] = exp(x_rsb*dFiter[Ns]);
		cumulativeSite[it] = exp(x_rsb*dFsite[Ns]);
		cumulativeLink[it] = exp(x_rsb*dFlink[Ns]);

		if (it>0){
			cumulativeSite[it] += cumulativeSite[it - 1];
			cumulativeIter[it] += cumulativeIter[it - 1];
			cumulativeLink[it] += cumulativeLink[it - 1];
		}
	}

	//reweigh
	for (int it = 0; it < Ns; it++){

		double myRand = (double)rand() / RAND_MAX * cumulativeIter[Ns - 1];
		int index = search_closest(cumulativeIter, myRand);

		appIter[it] = State[index];

		myRand = (double)rand() / RAND_MAX * cumulativeSite[Ns - 1];
		index = search_closest(cumulativeSite, myRand);

		appSite[it] = siteState[index];

		myRand = (double)rand() / RAND_MAX * cumulativeLink[Ns - 1];
		index = search_closest(cumulativeLink, myRand);

		appLink[it] = linkState[index];
		

	}

	for (int it = 0; it < Ns; it++){
		State[it] = appIter[it];
		siteState[it] = appSite[it];
		linkState[it] = appLink[it];

		
		
	}

}
\end{verbatim}

The same procedure is obtained at cluster level.

\begin{verbatim}
void mySite::reweigh(){

	for (int it = 0; it < Nc; it++){
		
		Cluster[it].reweigh();
		Cluster[it].evaluate();
			
		cumulativeIter[it] = exp(y_rsb*Cluster[it].Fiter);
		cumulativeSite[it] = exp(y_rsb*Cluster[it].Fsite);
		cumulativeLink[it] = exp(y_rsb*Cluster[it].Flink);

		if (it>0){
			cumulativeSite[it] += cumulativeSite[it - 1];
			cumulativeIter[it] += cumulativeIter[it - 1];
			cumulativeLink[it] += cumulativeLink[it - 1];
		}
	}

	//reweigh
	for (int it = 0; it < Nc; it++){

		double myRand = (double)rand() / RAND_MAX * cumulativeIter[Nc - 1];
		int index = search_closest(cumulativeIter, myRand);

		double myRandSite = (double)rand() / RAND_MAX * cumulativeSite[Nc - 1];
		int indexSite = search_closest(cumulativeSite, myRandSite);

		double myRandLink = (double)rand() / RAND_MAX * cumulativeLink[Nc - 1];
		int indexLink = search_closest(cumulativeLink, myRandLink);

		for (int jt = 0; jt < Ns; jt++){
		
			appCluster[it].State[jt] = Cluster[index].State[jt];
		
			appCluster[it].siteState[jt] = Cluster[index].siteState[jt];

			appCluster[it].linkState[jt] = Cluster[index].linkState[jt];
			

		}
	}

	for (int it = 0; it < Nc; it++){
		for (int jt = 0; jt < Ns; jt++){

			Cluster[it].State[jt] = appCluster[it].State[jt];
			
			Cluster[it].siteState[jt]=appCluster[it].siteState[jt];
			
			Cluster[it].linkState[jt] = appCluster[it].linkState[jt];
			
		}
	}

}
\end{verbatim}

\section{Evaluation of each observable}

The first routine reported is the one that averages the free energy at state level

\begin{verbatim}

void myCluster::evaluate(){
	
	double sum = 0, sumsite = 0, sumlink = 0;

	for (int it = 0; it < Ns; it++){
	sum     += exp(x_rsb*dFiter[it]);
	sumsite += exp(x_rsb*dFsite[it]);
	sumlink += exp(x_rsb*dFlink[it]);
	}

	Fsite = (1 / x_rsb)*log(sumsite / Ns);
	Fiter = (1 / x_rsb)*log(sum / Ns);
	Flink = (1 / x_rsb)*log(sumlink / Ns);
}
\end{verbatim}

The second routine reported is the one that uses the state-level values and averages over the cluster level.

\begin{verbatim}
void mySite::evaluate(){

	double beta = 1 / T;

	double sum = 0, sumsite = 0, sumlink = 0;

	for (int it = 0; it < Nc; it++){
		sum += exp(y_rsb*Cluster[it].Fiter);
		sumsite += exp(y_rsb*Cluster[it].Fsite);
		sumlink += exp(y_rsb*Cluster[it].Flink);
	}

	Fsite = (-1 / beta/ y_rsb)*log(sumsite / Nc);
	Fiter = (-1 / beta/ y_rsb)*log(sum / Nc);
	Flink = (-1 / beta / y_rsb)*log(sumlink / Nc);

	FREE_ENERGY = Fiter*(K + 1) / 2  - Fsite*(K - 1) / 2;
	FREE_ENERGY2 = Flink*(K + 1) / 2  - K*Fsite;
}

\end{verbatim}

The evaluation of the overlaps requires five 'for' cycles to be made. Since these cycles slow down the efficiency by a great amout, 
they are only used in a particular set of runs. Here i report the alternative version for the preceding function, used for the evaluation of the overlaps.


Among with that is reported the function that averages every observable on each site.

\begin{verbatim}
void mySystem::evaluate_observables(){

        //free energy
        MEAN = 0;
        double beta = 1 / T;
        
        q0 = 0; q1 = 0; q2 = 0;
        int c0 = 0, c1 = 0, c2 = 0;
        for (int it = 0; it < Nh; it++){

                MEAN += Site[it].FREE_ENERGY2;

        }
        
        for (int it = 0; it < Nh; it++){

                q0 += Site[it].q0;
                q1 += Site[it].q1;
                q2 += Site[it].q2;
        }
        
        MEAN /= Nh;
        q0 /= Nh; q1 /= Nh; q2 /= Nh;
}
\end{verbatim}
\section{System structure and memory reservation}
In this section the definition of the variables used is provided.

\begin{verbatim}
#pragma once
#include "defines.h"


class myCluster
{
public:
	myCluster();
	~myCluster();

	double State[Ns];
	double siteState[Ns];
	double linkState[Ns];

	double appSite[Ns];
	double appIter[Ns];
    double appLink[Ns];
    
	double dFiter[Ns];
	double dFsite[Ns];
	double dFlink[Ns];


	double Fiter;
	double Flink;
	double Fsite;

	double x_rsb;

	double cumulativeIter[Ns];
	double cumulativeSite[Ns];
    double cumulativeLink[Ns];
    
	void reweigh();
	void evaluate();

};

class mySite
{
public:
	mySite();
	~mySite();

	class myCluster Cluster[Nc];
	class myCluster appCluster[Nc];

	double y_rsb;

	double cumulativeSite[Nc];
	double cumulativeIter[Nc];
    double cumulativeLink[Ns];

	void reweigh();
	void evaluate();

	double Fiter;
	double Fsite;
	double Flink;

	double FREE_ENERGY;


};

class mySystem
{

public:
	mySystem();
	~mySystem();

	class mySite Site[Nh];
	class mySite *currSite;

	int time;
	int currSiteIndex;

	//methods
	void choose_sites();
	void write_file();

	void merge_fields();

	//neighbors
	int siteNeighs[K + 1], Jsite[K + 1];
	int linkNeighs[2 * K], Jlink[2 * K];
	int iterNeighs[K], Jiter[K];

	void evaluate_observables();
	fstream fileStream;

	double MEAN;
};

\end{verbatim} 