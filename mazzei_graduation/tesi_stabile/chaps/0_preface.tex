
\markboth{PREFACE}{PREFACE} 

This work will investigate the behaviour of the Bethe lattice spin glass in the low temperature phase.

The first part will give some generalities on spin glasses and, in particular, of the Bethe lattice spin glass.
In the second part of the elaborate  the previous studies made on this particular system will be summarized. This will include the first attempt of solution made by Bethe and Peierls nearly fifty years ago, valid only in the high temperature phase, and the approximate solution found by Mezard and Parisi much more recently, valid in low temperature phase as in high temperature one.

During the presentation will be assumed that the reader is somehow familiar with the mathematical tools that will be used hereafter. Those tools include the usual knowledge of elementary statistical mechanics and of disordered system, plus a brief insight into replica method and the much more favorable cavity method.
Replica calculations will not be developed in this paper due to the tricky aspect of the analytic continuation to zero replicas in the replica space. Since every result obtained with replica methods has later to be validated using cavity arguments, in this paper I preferred developing all the calculations directly in the cavity framework, altough it is usually much more difficult.
Each time an interesting result is obtained in the cavity framework, a paragraph is dedicated to the equivalence with the replica formalism. 

The third part will focus on an attempt of two step replica symmetry breaking, or, to be precise, the cavity equivalent of it. This will be done using the solution found by Mezard and Parisi as a starting point. Among with it an algorithm constructed to evaluate the main two step RSB observables is ideated and generated. The algorithm takes advantage of the hierarchical structure of the replica symmetry breaking framework. In line of principle the algorithm may be extended to higher levels of RSB, however both the computation effort and the memory reservation increase exponentially in the RSB level.
The last chapter will focus on the results, finding the correct parameters for two step replica symmetry breaking, among with a computation of the replica overlaps. Results are then compared to Monte Carlo simulation of the system using a parallel tempering.
The algorithm will be reported in the Appendix, together with some mathematical tools useful during the work. 

The elaborate refers mainly to a paper published by M. Mezard and G. Parisi in 2001 \cite{bethe}, that will be the beacon light of the whole work presented. 

%Despite the apparent simplicity, Ising models have been and are currently widely studied among physicists. Roughly speaking, an Ising model is a collection of dicotomic variables $\sigma_i , i \in N$ (this means that can have only two values) interacting on couples with energy $J_{ij}\sigma_i\sigma_j$ embedded in an external field adding energy $H_i\sigma_i$ to each site. The energy of the system is the sum of the interaction and site contributions. The original Ising model was defined on a regular square lattice, $\sigma = \{\pm 1\}$, and $J_{ij} = -1$ (he had in mind a crystal with ferromagnetic spin interaction in each site). He provided a correct solution on one-dimensional chain, predicting no phase transition, and a wrong solution on bi-dimensional lattice. The correct solution has been found by Onsager (19??)[?]. An analytic solution in three dimensions of the regular lattice Ising model is still missing.

%Since the original Ising's work on ferromagnets, this model underwent a lot of generalizations and a huge variety 
%of studies have been made on it. It has been investigated in various forms, varying within a number of lattice morphologies and interaction schemes. 

%The number of direct applications of this phenomenological model is huge. It has been used as a model for magnets, gases and much more other physical systems[][][]. It also has been used in image processing[], disease-spreading analysis[], political elections models, just to mention a few.

%However recently the most important studies verted on spin-glasses. Is indeed in spin glass systems that major results have been obtained. A totally new behaviour[] has been predicted in such systems below a characteristic temperature. Spin glasses exibith non ergodic behaviour and by consequence the coexistence of many metastable states. A number of theoretical methods has been developed[][] and research on spin glasses is nowadays still in progress.


