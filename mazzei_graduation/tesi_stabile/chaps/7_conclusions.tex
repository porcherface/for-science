\chapter{Conclusion of the work}

In this conclusive chapter I will regroup the results obtained. A small section will be dedicated to the possibility of extending the method described to higher levels of RSB.

The main task of this study was to give a quantitative evaluation of the two step RSB  relevant observables (with a particular attention to $q_0$, $q_1$ and $q_2$. It has been shown that results are compatible with two step broken symmetry. Similarly to what happens in the SK model, replica symmetry is broken in a complicated way.
It is possible to argue that RSB is present at any level. In line of principle the algorithm used can be extended to highers levels of RSB. Since it is possible to reach high levels of algorithmical complexity with modern cluster computer (I personally believe that a good result may be obtained up to four RSB levels, though the utility of such computation is not obvious), it would be interesting to develop a recursive version of the algorithm described in chapter 5, capable of reaching any desired level of RSB.

A possible future investigation can be done on a way to increase the algorithm's computational efficiency. This might be mandatory if one wants to reach higher levels of RSB. Since the algorithm's complexity, as described in this work, has a complexity equal to $N_h N_c {N_s} ^ 2$, a good estimation of the relevant observables is complicated even in this two step RSB case. However, although the low precision of the runs performed, the algorithm has been capable to achieve good enough results.

The results obtained may also give us a hint on the shape of the Parisi overlap function $q(x)$. As we can see from the values of $x_i$ (where $i$ for now can only be chosen between $\{ s, C\}$), these two tend to be more or less near to zero. This indicates a fast increasing $q(x)$ in the left region of the interval $x \app [0,1]$, similarly to a result obtained on other spin glass systems\cite{function}.
