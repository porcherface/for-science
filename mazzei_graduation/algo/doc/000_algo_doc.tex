%% Based on a TeXnicCenter-Template by Gyorgy SZEIDL.
%%%%%%%%%%%%%%%%%%%%%%%%%%%%%%%%%%%%%%%%%%%%%%%%%%%%%%%%%%%%%

%----------------------------------------------------------
%
\documentclass{amsbook}%
%
%----------------------------------------------------------
% This is a sample document for the AMS LaTeX Book or Monograph Class
% Class options
%       --  Body text point size:
%                        8pt, 9pt, 10pt (default), 11pt, 12pt
%       --  Paper size:  letterpaper (8.5x11 inch, default), a4paper
%       --  Orientation:
 portrait(default), landscape
%       --  Print side:  oneside, twoside (default)
%       --  Quality:     final(default), draft
%       --  Title page:  titlepage, notitlepage
%       --  Start chapter on left:
%                        openright (no, default), openany
%       --  Columns:     onecolumn (default), twocolumn
%       --  Omit extra math features:
%                        nomath
%       --  AMS fonts (noamasfonts available):
%                        noamsfonts
%       --  PSAMSfonts (fewer AMSfontsizes)
%                        psamsfonts
%       --  Equation numbering (equation numbers on the left is the default)
%                        leqno (default), reqno
%       --  Equation centering (equations centered is the default)
%                        centeredtags (default}, tbtags (top, bottom)
%       --  Displayed equations (centered is the default)
%                        fleqn (flush left)
% For instance the command
%          \documentclass[a4paper,12p,reqno]{amsbook}
% ensures that the paper size is a4, fonts are typeset at the size 12p
% and the equation numbers are on the right side.
%
\usepackage{amsmath}%
\usepackage{amsfonts}%
\usepackage{amssymb}%
\usepackage{graphicx}
\usepackage{hyperref}
\usepackage{color}
\usepackage{datetime}

\DeclareMathOperator{\Tr}{Tr}
\DeclareMathOperator{\atanh}{atanh}
%----------------------------------------------------------
\theoremstyle{plain}
\newtheorem{acknowledgement}{Acknowledgement}
\newtheorem{algorithm}{Algorithm}
\newtheorem{axiom}{Axiom}
\newtheorem{case}{Case}
\newtheorem{claim}{Claim}
\newtheorem{conclusion}{Conclusion}
\newtheorem{condition}{Condition}
\newtheorem{conjecture}{Conjecture}
\newtheorem{corollary}{Corollary}
\newtheorem{criterion}{Criterion}
\newtheorem{definition}{Definition}
\newtheorem{example}{Example}
\newtheorem{exercise}{Exercise}
\newtheorem{lemma}{Lemma}
\newtheorem{notation}{Notation}
\newtheorem{problem}{Problem}
\newtheorem{proposition}{Proposition}
\newtheorem{remark}{Remark}
\newtheorem{solution}{Solution}
\newtheorem{summary}{Summary}
\newtheorem{theorem}{Theorem}
\numberwithin{equation}{section}

\newcommand{\HRule}{\rule{\linewidth}{0.5mm}}
%-----------------------------------------------------------
\begin{document}
\frontmatter

\begin{titlepage}
\begin{center}

% Upper part of the page. The '~' is needed because \\
% only works if a paragraph has started.
\begin{figure}

\centerline{\includegraphics{img/dip-logo-uff.png}}
\end{figure}

\textsc{\LARGE La Sapienza, University of Rome}\\[1.5cm]

\textsc{\Large Master degree final discussion}\\[0.5cm]

% Title
\HRule \\[0.4cm]
{ \huge \bfseries A two step RSB algorithm on Bethe lattice spin glass - Algorithm technical description\\[0.4cm] }

\HRule \\[1.5cm]

% Author and supervisor
\begin{minipage}{0.4\textwidth}
\begin{flushleft} \large
\emph{Author:}\\
Andrea \textsc{Mazzei}
\end{flushleft}
\end{minipage}
\begin{minipage}{0.4\textwidth}
\begin{flushright} \large
\emph{Supervisor:} \\
Prof.~Giorgio \textsc{Parisi}
\end{flushright}


\date{\large \today}

\end{minipage}

% Bottom of the page
\vfill

\end{center}
\end{titlepage} 

\subjclass{La Sapienza, University of Rome, Theoretical physics master degree final discussion}

\maketitle
\tableofcontents

\mainmatter

\chapter{Introduction}

This document provides a detailed description of the algorithm used to evaluate the two-step RSB observables in the Bethe lattice spin glass. The approach to the problem, the detail of the mathematical procedures and each other information not strictly related to the computer implementation of the algorithm are reported in the main work.

The algorithm reported has been developed in C++. 


\chapter{Algorithm description}

The population 


\input{100_others.tex}



\end{document}
